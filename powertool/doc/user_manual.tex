\ifx \inmaxwitbook \undefined
\documentclass[a4paper,11pt,oneside]{article}

\usepackage{float}
\usepackage{graphicx}
\usepackage{longtable}
\usepackage{multirow}
\usepackage{CJK}
\usepackage{listings}
\usepackage{xcolor}
\usepackage{fancyhdr}
\usepackage{indentfirst}
\usepackage{amsmath}
\usepackage{wasysym}
\usepackage{rotating}
\usepackage[
	dvipdfm,
	CJKbookmarks=true,
	bookmarksnumbered=true,
	bookmarksopen=false,
	colorlinks,
	linkcolor=black
	]{hyperref}

\renewcommand{\footnotesize}{}
\renewcommand{\footrulewidth}{0pt}

\setlength{\unitlength}{2em} % for the picture environment

% 4 levels of headings
\newcommand{\headA}[1]{{\huge \vspace{1ex} \bf \hfill #1 \vspace{1ex}\hfill}}
\newcommand{\headB}[1]{{\Large \vspace{1ex} \bf #1 \vspace{1ex} }}
\newcommand{\headC}[1]{{\large \vspace{1ex} \bf #1 \vspace{1ex} }}
\newcommand{\headD}[1]{{\bf \vspace{1ex} #1 \vspace{1ex} }}

\newcommand{\pushin}{\hspace*{1em}}

\lstdefinelanguage{mwshell}{
	morekeywords={if,then,elif,else,fi,for,in,while,do,done},
	morecomment=[l]{\#\ },
}

\lstset{
	numbers=left,
	%framexleftmargin=10mm,
	%frame=none,
	%frame=shadowbox,
	%rulesepcolor=\color{blue},
	backgroundcolor=\color[RGB]{230,230,230},
	tabsize=4,
	breaklines=true,
	keywordstyle=\bf\color{blue},
	%identifierstyle=\bf,
	%numberstyle=\color[RGB]{0,192,192},
	commentstyle=\it\color[RGB]{0,96,96},
	%stringstyle=\rmfamily\slshape\color[RGB]{128,0,0},
	showstringspaces=false,
	lineskip=0.5pt,
	basicstyle=\ttfamily\small
}

\setlength{\parindent}{2em}

\usepackage[body={14cm,22cm},bindingoffset=2cm]{geometry}

\begin{CJK*}{UTF8}{gbsn}
\title{MaxWit PowerTool Users' Manual}
\author{胡笑翔\\Conke Hu \and 晏成\\Rouchel Yan}
\end{CJK*}

%\renewcommand{\baselinestretch}{1.3}
\linespread{1.3}

\begin{document}
\AtBeginDvi{\special{pdf:tounicode UTF8-UCS2}}
\begin{CJK*}{UTF8}{gbsn}
\CJKtilde
\maketitle
\tableofcontents

\newpage
\fi

\section{系统安装}
准备工作主要做三件事情:磁盘分区、安装~Windows~以及制作~Ubuntu~系统安装盘。不过在此之前,你需要先~review~一下爱机的硬件配置。

\subsection{硬件配置(推荐)}

\begin{tabular}{|c|c|l|}
\hline
\textbf{配件} & \textbf{型\ \ 号} & \textbf{\ \ \ \ \ \ \ \ \ \ \ 备\ \ 注} \\ \hline
CPU & Core i5/i7 & 4 cores\\ \hline
RAM & 4G DDR3 & 2~G for native Linux, 2~G for VM OS \\ \hline
GPU & Intel~集成 & Better support than A or N card\\ \hline
HDD & 不小于~320G & = 4 + 20 + 60 + (160 + 40 + 20) \\ \hline
Wi-Fi & Intel * & Better support than other vendors \\ \hline
\end{tabular}

\subsection{磁盘分区}
用~PartitionsMagic~(~Windows~系统安装盘自带该工具(~PQ~),可先进入~Windows~系统安装界面使用此工具)图形分区工具,将硬盘分成几个区,用来装双系统(~Windows~和~Ubuntu~,不同的系统安装在同一个硬盘的不同分区上),具体分区可自定,但推荐使用以下规范来分区。~PartitionsMagic~不能按``ext4''文件系统格式分区,因此在这里先按~``ext3''~文件系统格式分区,在安装~Ubuntu~系统是改变分区类型即可。

以容量为~320G~的硬盘为例,共分4个~Primary Partitions~(主分区)和2个~Extended Partitions~(扩展分区):1$\sim$4为主分区,5$\sim$6为扩展分区。具体分区情况见下两个表格。其中``用途''一列若不明白可暂时跳过,后面很快就会懂:)。

\begin{centering}
\begin{tabular}{|c|c|c|c|l|}
\hline
%\multicolumn{1}{|c|}{序号} & \multicolumn{1}{|c|}{类 型} & \multicolumn{1}{|c|}{大 小} & \multicolumn{1}{|c|}{用  途} & \multicolumn{1}{|c|}{mount point} \\ \hline
\textbf{序号} & \textbf{类\quad 型} & \textbf{大小}  & \textbf{mount} & \textbf{\quad\quad\quad\quad用\quad\quad 途} \\ \hline
1 & SWAP & 4G & (NULL) & Swap for Linux, also for Win 7 \\ \hline
2 & Ext3 & 20G & / & Linux rootfs (Ubuntu~系统分区) \\ \hline
3 & NTFS & 60G & (NULL) & Windows 7 (or Mac OS)~系统分区 \\ \hline
4 & 扩展分区 & 220G & (NULL) & 扩展出第~5、6、7~三个分区 \\ \hline
%\end{tabular}
%
%Extended Partitions(扩展分区):
%
%\begin{tabular}{|c|p{3cm}|c|c|c|}
\hline
%\multicolumn{1}{|c|}{序号} & \multicolumn{1}{|c|}{类 型} & \multicolumn{1}{|c|}{大 小} & \multicolumn{1}{|c|}{用  途} & \multicolumn{1}{|c|}{mount point} \\ \hline
%\textbf{序号} & \textbf{类\ 型} & \textbf{大\ 小} & \textbf{用\ \ \ \ 途} & \textbf{Mount Point} \\ \hline
5 & Ext3 & 160G & /maxwit & Linux~数据分区~(\textbf{as large as possible!}) \\\hline
6 & NTFS & 40G & /data & Windows~数据分区(D~盘)\\\hline
7 & Ext3 & 20G & (NULL) & 玩玩其他~Linux~发行版,兼作备用系统\\\hline
\end{tabular}
\\

\centerline{表1,320G~硬盘系统分区(推荐)}
\end{centering}

\subsection{Install Ubuntu 10.10}
我们选择~Ubuntu 10.10~作为主要工作平台。

第八步,进入~``who are you?''~用户配置界面,按如下说明填写:
\indent	\begin{description}\setlength{\itemsep}{-\itemsep}
	\item [``Your name:''] 用户名称,请务必采用``英文名+中文姓''的形式!如``Jet Li''。
	\item [``Your computer's name''] 填``maxwit.com''
	\item [``Pick a username''] 系统用户名,已自动填为``jet'',不用改
	\item [``Choose a password''] 登陆密码,如``maxwit''
	\item [``Confirm your password''] 密码确认。
	\end{description}

\section{PowerTool:帮你轻松构建嵌入式开发环境}

\subsection{运行~PowerTool}

执行如下命令即可:
\begin{lstlisting}
$ sudo apt-get install wget
$ cd
$ wget http://192.168.0.2/powertool.tar.gz
$ tar xvf powertool.tar.gz
$ cd powertool
$ ./build
\end{lstlisting}

逐行讲解:

第1行,先请确保系统中已经安装了~wget~(一个被企业及开源社区广为采用的项目管理工具)。若未安装,则现在安装之:

第2行,从server~上~checkout~出~powertool~的~source code:

建议暂时先把~powertool~放在当前~user~的~HOME~目录下(上面~wget clone~之前先执行下~``cd'')。

提示:若~server~上~powertool~有~update,可通过执行如下命令使之与~server~保持同行:

命令,即可获取最新的~powertool~源码:)。

第3行,运行~PowerTool

进入~powertool~目录,执行~build~脚本即可:

这个过程可能需要较长时间,强烈建议盯着你的screen,观察整个执行过程中的每个细节!甚至是每个~deb package~的名字!

恭喜你,现在你已搭建好一个非常漂亮的嵌入式~Linux~开发环境!怎么样,很~easy~吗?\smiley

\subsection{PowerTool~完成了哪些工作?}
\begin{enumerate}
\item 配置~software source(软件源)
\item 安装必备的系统工具,包括:\\
VIM, 
\item 安装开发工具及常用开发库
\item 配置~E-mail client
\item 配置~git client
\item 安装\LaTeX
\item populate ``/maxwit''目录树
\item 检测并安装设备驱动(suspending)
\item ...(等你去发现)
\end{enumerate}

\section{掌握你的~Linux Host~环境}
\subsection{测试~GNU Toolchain: gcc, g++}

\subsection{使用~Eclipse IDE}

\subsection{使用~邮件客户端}

\subsection{测试显卡的~3D Performance}

\subsection{测试~File System}

\ifx \inmaxwitbook \undefined
\clearpage
\end{CJK*}
\end{document}
\fi
